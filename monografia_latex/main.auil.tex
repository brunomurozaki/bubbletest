% Document type and layout.
\documentclass[12pt,a4paper,espaco=umemeio,noindentfirst,oneside,openany,tocpage=plain,pnumromarab,ruledheader,time,anapcustomindent]{sty/abnt}

% Brazilian portuguese language stuff (acents, etc.).
\usepackage[brazil]{babel}

% Encoding (or charset).
% This is a feature depending on the configuration of the text editor
% or the particular system which you are using.
\usepackage[utf8]{inputenc}
% If you experience some problem with the charset encoding (weird
% simbols being displayed) try to change the option utf8 below to
% latin1 (which sets the charset to ISO-8859-1 encoding).
%\usepackage[latin1]{inputenc}

% Load the package abncite, for bibliographical citations into the
% text body. The option call=authordata below sets the layout of
% citations to Autor-Date style.
%\usepackage[call=authordata,order=alphabetic,recuo=0.5cm,abnt-emphasize=bf,
%abnt-and-type=&,abnt-etal-list=3,abnt-etal-cite=3]{abntcite}
% but doesn't work. Use this instead:
\usepackage[call=numeric,order=alphabetic,recuo=0.5cm,abnt-emphasize=bf,
abnt-and-type=&,abnt-etal-list=3,abnt-etal-cite=3]{abntcite}
% If you want to use numbers, comment out the line below.
%\usepackage[call=numeric,order=citation,recuo=0.5cm,abnt-emphasize=bf,
%abnt-and-type=&,abnt-etal-list=3,abnt-etal-cite=3,
%call=brackets,biblabel=show,biblabel=brackets]{sty/abntcite}

%%%%%%%%%%%%%%%%%%%%%%%%%%%%%%%%%%%%%%%%%%%%%%%%%%%%%%%%%%%%%%%%%%%%%%

% PDF output and additional related settings.
\usepackage{sty/pdfcompat}

% Additional ABNTex definitions.
\usepackage[disable=copyright,disable=biblabel]{sty/ach2017}

% This package seems to conflicts with logo-each above, if loaded
% before that package. Therefore, put it here.
\usepackage[T1]{fontenc}

% To type URL with linebreak at special characters.
\usepackage{url}

\usepackage[unicode=true,bookmarks=true,bookmarksnumbered=false,
bookmarksopen=true,bookmarksopenlevel=1,breaklinks=false,
pdfborder={0 0 1},backref=section,colorlinks=false]{hyperref}

% Improved and customizable hyphenation patterns.
\usepackage{hyphenat}
\hyphenation{pe-rio-do res-pon-sá-vel}

%%%%%%%%%%%%%%%%%%%%%%%%%%%%%%%%%%%%%%%%%%%%%%%%%%%%%%%%%%%%%%%%%%%%%%

\renewcommand{\ABNTchapterfont}{\fontfamily{ptm}\bfseries\selectfont}
\renewcommand{\tituloformat}{\huge\ABNTchapterfont}

%%%%%%%%%%%%%%%%%%%%%%%%%%%%%%%%%%%%%%%%%%%%%%%%%%%%%%%%%%%%%%%%%%%%%%

% Define the base path for the graphic files, anchored to the
% directory in which this .sty file is placed (recommended).
%\graphicspath{{./pictures/}}

% Produce lists of symbols as in nomenclature.
\usepackage{nomencl}
% the following is useful when we have the old nomencl.sty package
\providecommand{\printnomenclature}{\printglossary}
\providecommand{\makenomenclature}{\makeglossary}
\makenomenclature
\renewcommand{\nomname}{Glossário}
\renewcommand{\nomlabel}[1]{{\bf #1}:}

% Delicatessen pour (LaTeX) connaisseurs.
\usepackage{lastpage}

% Add background color to tables.
\usepackage{colortbl}

%\usepackage{lscape}

%%%%%%%%%%%%%%%%%%%%%%%%%%%%%%%%%%%%%%%%%%%%%%%%%%%%%%%%%%%%%%%%%%%%%%
% User defined gadgets.

\newcommand{\up}[1]{\raisebox{1.4ex}[0pt]{#1}}
\newcommand{\struthis}[1]{\rule{0pt}{4ex}\raisebox{1ex}{#1}}

\definecolor{buttonblue}{rgb}{0.25,0.40,0.56}
\newcommand{\buttonblue}[1]{\colorbox{buttonblue}{\textcolor{white}{\sf\footnotesize #1}}}
\newcommand{\buttonred}[1]{\colorbox{red}{\textcolor{white}{\sf\footnotesize #1}}}

%%%%%%%%%%%%%%%%%%%%%%%%%%%%%%%%%%%%%%%%%%%%%%%%%%%%%%%%%%%%%%%%%%%%%%

\renewcommand{\thesection}{\arabic{section}}


\begin{document}

%%%%%%%%%%%%%%%%%%%%%%%%%%%%%%%%%%%%%%%%%%%%%%%%%%%%%%%%%%%%%%%%%%%%%%
% Fill the fields below, in order to provide the informations for the
% pre-textual elements.

\instituicao{Universidade de São Paulo\\Escola de Artes, Ciências e
  Humanidades}

\autor[Rocha, Larissa Teles da]{Larissa Teles da Rocha}

\titulo{Tripulações aéreas e hotéis: os critérios de escolha
  utilizados por companhias aéreas e as adaptações realizadas pelos
  hoteis em seus produtos e serviços}

\local{São Paulo}

\data{Fevereiro de 2010}

\comentario{Modelo de monografia para ser apresentada à Escola de
  Artes, Ciências e Humanidades da Universidade de São Paulo, como
  parte dos requisitos exigidos para aprovação na disciplina
  \disciplinaname, do curso de Bacharelado em Sistemas de Informação.}

\area[Modalidade:]{Trabalho de graduação curto -- 1 semestre.}

\orientador{Prof. Dr. Renato Braz Oliveira de Seixas}

\defesadata{29 de fevereiro de 2010}

\bancaone{Profª Drª Ariane Machado Lima}
\bancatwo{Prof. Dr. Antonio Luciano Digiampietri}

\adviser[Seixas, Renato Braz Oliveira de]{Renato Braz Oliveira de
  Seixas}

\paginas{\pageref{LastPage} p. : il.}

\catalogtop{{\ABNTtitulodata} / {\ABNTautordata} ; orientação de
  \ABNTadviserdata. -- {\ABNTlocaldata} : Universidade de São Paulo,
  Escola de Artes, Ciências e Humanidades, \ABNTdatadata.}

\catalogmed{Monografia apresentada para Conclusão de Curso de Lazer e
  Turismo -- Escola de Artes, Ciências e Humanidades da Universidade
  de São Paulo.} 

\catalogbottom{1. Hotelaria. 2. Empresas de transportes
  turísticos. I. \ABNTvaradviserdata, orient. II. Título.}

\catalogcdd{CDD 22.ed. -- 910.46}

%%%%%%%%%%%%%%%%%%%%%%%%%%%%%%%%%%%%%%%%%%%%%%%%%%%%%%%%%%%%%%%%%%%%%%
% Pre-Textual elements without title and without numeric indexation
% (5.3.4).

% Mandatory (4.1.1 Capa).
\capa

% Mandatory (4.1.3 Folha de rosto).
\folhaderosto
\folhaderostoreverso

% Mandatory (4.1.5 Folha de aprovação).
\folhadeaprovacao

%%%%%%%%%%%%%%%%%%%%%%%%%%%%%%%%%%%%%%%%%%%%%%%%%%%%%%%%%%%%%%%%%%%%%%
% Pre-Textual elements without numeric indexation (5.3.3). The title
% of these elements must be centralized, as stated in NBR 6024.

% Optional (4.1.6 Dedicatória).
\pretextualchapter{Dedicatória} % Title.

bla bla bla bla bla bla bla bla bla bla bla bla

bla bla bla bla bla bla bla bla bla bla bla bla

% Optional (4.1.7 Agradecimentos).
\pretextualchapter{Agradecimentos} % Title.

bla bla bla bla bla bla bla bla bla bla bla bla

bla bla bla bla bla bla bla bla bla bla bla bla

% Mandatory (4.1.9 Resumo na língua vernácula).
\palavraschave{ABNT, norma NBR 6028, elementos pre-textuais, resumo.}
\begin{resumo}
Elemento obrigatório constituido de uma sequência de frases concisas e
objetivas e não de uma simples enumeração de tópicos, não
ultrapassando 500 palabras, seguido logo abaixo, das palavras
representativas do conteúdo do trabalho, isto é, palabras-chave e/ou
descritores, conforme a ABNT NBR 6028. As diretrizes da USP também
recomendam que o resumo seja redigido em parágrafo único, embora essa
especificação não conste na norma NBR 14724.

\makekeywords
\end{resumo}

% Mandatory (4.1.10 Resumo em língua estrangeira).
\renewcommand{\ABNTabstractname}{Résumé}
\palavraschave[Keywords:]{ABNT, norm NBR 6028, pre-textual elements, abstract.}
\begin{abstract}
Mandatory element, with the same features of the Abstract in the
native language, typed on a separate sheet (in English
\emph{Abstract}, in Spanish \emph{Resumen}, in French \emph{Résumé},
for instance). It must be followed by the representative words of the
work subject, i.e., keywords and/or descriptors in the given
language. Article 99 of Regimento do Pós-Graduação da USP states that
the Abstract must be redacted in English language, for divulgation
purpouses. Exceptions to this rule could be allowed at criterion of
the CPG of each faculty.

\makekeywords
\end{abstract}

% Optional (4.1.11 Lista de ilustrações).
\listadefiguras

% Optional (4.1.11 Lista de tabelas).
\listadetabelas

% Optional(?): Glossary.
\printnomenclature

% Mandatory (4.1.15 Sumário).
\sumario

%%%%%%%%%%%%%%%%%%%%%%%%%%%%%%%%%%%%%%%%%%%%%%%%%%%%%%%%%%%%%%%%%%%%%%
% Textual elements.

%\cleardoublepage
%%\clearpage
%\chapter*{}
%\section{Título de seção isolada sem número de capítulo}

\chapter{Introdução} % Chapter title.

No Glossário as palavras deverão aparecer em ordem alfabética. A
palavra ou sigla a ser definida grafada em negrito seguida de dois
pontos (:), e em seguida a definição é escrita sem negrito. Por
exemplo, veja como a sigla GLCE
\nomenclature{GLCE}{Gramática Livre de Contexto Estocástica –
  gramática livre de contexto com uma distribuição de probabilidades
  sobre as produções com o mesmo lado esquerdo.}
aparece no Glossário.

%%%%%%%%%%%%%%%%%%%%%%%%%%%%%%%%%%%%%%%%%%%%%%%%%%%%%%%%%%%%%%%%%%%%%%

\chapter{Objetivos}

\section{Objetivo Geral}

bla bla bla bla bla bla bla bla bla bla bla bla

\section{Objetivos Específicos}

bla bla bla bla bla bla bla bla bla bla bla bla

%%%%%%%%%%%%%%%%%%%%%%%%%%%%%%%%%%%%%%%%%%%%%%%%%%%%%%%%%%%%%%%%%%%%%%

\chapter{Revisão bibliográfica}

Exemplo de citação~\cite{Pressman:2006}. Outro exemplo \cite{Levy:1999}.

%%%%%%%%%%%%%%%%%%%%%%%%%%%%%%%%%%%%%%%%%%%%%%%%%%%%%%%%%%%%%%%%%%%%%%

\chapter{Metodologia}
Neste capítulo mostra-se um exemplo de cronograma:

\begin{table}[h]
\centering
\begin{tabular}{|p{5cm}|>{\centering}p{15mm}|>{\centering}p{15mm}|>{\centering}p{15mm}|>{\centering}p{15mm}|>{\centering}p{15mm}|}
\hline
\multicolumn{1}{|c|}{Atividade} & Fevereiro & Março & Abril & Maio & Junho \tabularnewline
\hline
\hline
\struthis{Plano de Atividades} & \cellcolor[gray]{.4} & \cellcolor[gray]{.4} &  &  & \tabularnewline
\hline
\struthis{Diagrama do banco de dados} &  & \cellcolor[gray]{.4} &  &  & \tabularnewline
\hline
Entrega do Plano de Atividades &  & \cellcolor[gray]{.4} {\color{white}Dia
12} &  &  & \tabularnewline
\hline
\struthis{Diagramas UML} &  & \cellcolor[gray]{.4} &  &  & \tabularnewline
\hline
Instalação dos servidores de banco de dados e WWW &  & \cellcolor[gray]{.4} &  &  & \tabularnewline
\hline
Desenvolvimento das classes da aplicação &  & \cellcolor[gray]{.4} & \cellcolor[gray]{.4} & \cellcolor[gray]{.4} & \tabularnewline
\hline
Apresentação sobre o andamento do trabalho &  &  &
\cellcolor[gray]{.4} {\color{white}Dia 16} &  & \tabularnewline
\hline
Desenvolvimento das telas e interface da aplicação &  & \cellcolor[gray]{.4} & \cellcolor[gray]{.4} & \cellcolor[gray]{.4} & \tabularnewline
\hline
Desenvolvimento do Trabalho Monográfico &  &  &  & \cellcolor[gray]{.4} & \cellcolor[gray]{.4} \tabularnewline
\hline
Entrega do trabalho monográfico &  &  &  &  & \cellcolor[gray]{.4} {\color{white}Dia 11} \tabularnewline
\hline
\end{tabular}
\caption{Cronograma do desenvolvimento do projeto.}
\end{table}


%%%%%%%%%%%%%%%%%%%%%%%%%%%%%%%%%%%%%%%%%%%%%%%%%%%%%%%%%%%%%%%%%%%%%%

\chapter{Resultados}
Para cadastrar um novo usuário, deve-se preencher os dados e clicar no
botão \buttonblue{\quad Enviar\quad}. Para excluir um usuário do
sistema, deve-se clicar no botão \buttonred{\quad Excluir\quad} na
linha correspondente ao usuário que deseja-se excluir.

%%%%%%%%%%%%%%%%%%%%%%%%%%%%%%%%%%%%%%%%%%%%%%%%%%%%%%%%%%%%%%%%%%%%%%

\chapter{Discussão}

bla bla bla bla bla bla bla bla bla bla bla bla

bla bla bla bla bla bla bla bla bla bla bla bla

%%%%%%%%%%%%%%%%%%%%%%%%%%%%%%%%%%%%%%%%%%%%%%%%%%%%%%%%%%%%%%%%%%%%%%

\chapter{Conclusão}

bla bla bla bla bla bla bla bla bla bla bla bla

bla bla bla bla bla bla bla bla bla bla bla bla

%%%%%%%%%%%%%%%%%%%%%%%%%%%%%%%%%%%%%%%%%%%%%%%%%%%%%%%%%%%%%%%%%%%%%%
% Post-textual element: Bibliographical references (mandatory).

\bibliographystyle{bst/abnt-alf}
%\bibliographystyle{bst/abnt-num}

% You can use this:
%\bibliography{bib/example}
% or the stuff below, but NOT both simultaneously.
\renewcommand{\bibfile}{bib/example}
\renewcommand{\nocitebibfile}{bib/nocite}
\renewcommand{\nocitelist}{Apache,WAI}
\referencesandbibliography

%%%%%%%%%%%%%%%%%%%%%%%%%%%%%%%%%%%%%%%%%%%%%%%%%%%%%%%%%%%%%%%%%%%%%%
% Post-textual element: Appendix (optional).

% From now on, the chapters are no longer referenciated by numbers,
% but as "Appendix", and with an alpha-numeric identifier.
\appendix


\chapter{Título do Apêndice A}

bla bla bla bla bla bla bla bla bla bla bla bla

bla bla bla bla bla bla bla bla bla bla bla bla

%%%%%%%%%%%%%%%%%%%%%%%%%%%%%%%%%%%%%%%%%%%%%%%%%%%%%%%%%%%%%%%%%%%%%%

\chapter{Título do Apêndice B}

bla bla bla bla bla bla bla bla bla bla bla bla

bla bla bla bla bla bla bla bla bla bla bla bla

%%%%%%%%%%%%%%%%%%%%%%%%%%%%%%%%%%%%%%%%%%%%%%%%%%%%%%%%%%%%%%%%%%%%%%
% Annex.

% From now on, the chapters are referenciated as "Annex", and with an
% alpha-numeric identifier.
\annex


\chapter{Título do Anexo A}

bla bla bla bla bla bla bla bla bla bla bla bla

bla bla bla bla bla bla bla bla bla bla bla bla

%%%%%%%%%%%%%%%%%%%%%%%%%%%%%%%%%%%%%%%%%%%%%%%%%%%%%%%%%%%%%%%%%%%%%%

\chapter{Título do Anexo B}

bla bla bla bla bla bla bla bla bla bla bla bla

bla bla bla bla bla bla bla bla bla bla bla bla

%%%%%%%%%%%%%%%%%%%%%%%%%%%%%%%%%%%%%%%%%%%%%%%%%%%%%%%%%%%%%%%%%%%%%%

\end{document}
